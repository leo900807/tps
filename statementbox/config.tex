\documentclass[10pt, a4paper]{article}
\usepackage[top=3cm, bottom=3cm, left=2cm, right=2cm]{geometry} % 上下左右距離邊緣 2cm
\usepackage{amsmath, amsthm, amssymb}   % AMS 數學環境
\usepackage{enumitem}   % enumerate itemize
\usepackage{xeCJK}      % 中文
\usepackage{fontspec}   % 設定字體
\usepackage{anyfontsize}    % \fontsize{}{}
\usepackage{graphicx}   % 圖片
\usepackage{caption}    % for \caption*
\usepackage{fancyhdr}   % 頁首頁尾
\usepackage{titlesec}   % 設定 section 字體、\titlespacing
\usepackage{tabularx, makecell} % 加強版表格
\usepackage{hyperref}   % 超連結
\usepackage{xcolor}     % 文字顏色
\usepackage{ulem}       % 底線
\usepackage{eso-pic}    % 圖標
\usepackage{booktabs}   % 表格線條
\usepackage{listings}   % 程式碼上色
\usepackage{indentfirst}    % 縮排
\usepackage{setspace}   % 行間距
\usepackage{seqsplit}   % 換行
\usepackage{fancyvrb}
\usepackage[T1]{fontenc}

% 設定 section, subsection 字體
\titleformat*{\section}{\fontsize{15.5}{15.5}\selectfont\bfseries}
\titleformat*{\subsection}{\fontsize{13}{13}\selectfont\bfseries}

% 設定 math mode 字體
\DeclareMathSizes{10}{12}{7}{5}
\setlength{\mathsurround}{1.0pt}

\XeTeXlinebreaklocale "zh"  % 中文自動換行
\XeTeXlinebreakskip = 0pt plus 1pt  % 文字的彈性間距
\linespread{1.35}\selectfont  % 行距
\setlength{\parskip}{1.8ex} % 段落間距
\setitemize[1]{itemsep=-4pt,partopsep=0pt,parsep=1ex,topsep=5pt} % itemize 間距
\setenumerate[1]{itemsep=0pt,partopsep=0pt,parsep=\parskip,topsep=5pt} % enumerate 間距
\setlength{\parindent}{2em} % 縮排

% 設定 section 左邊的 hspace 以及上下的 vspace
\titlespacing\section{0pt}{12pt plus 4pt minus 2pt}{10pt plus 2pt minus 2pt}
\titlespacing\subsection{0pt}{16pt plus 4pt minus 2pt}{10pt plus 2pt minus 2pt}

% 設定中文字型
\setCJKmainfont[AutoFakeBold=4]{Noto Serif CJK TC} % 一般中文字體
\newcommand{\chfont}{SourceHanSansTW-Bold.ttf} % 標題字體
\setCJKfamilyfont{\chfont}{\chfont}[AutoFakeBold = 3]

% 設定英文字型
% \setmainfont{Times New Roman}
% \usepackage{newtxtext}
\setmonofont{Monaco.ttf}

\newcommand{\myhref}[2]{\textcolor{blue}{\href{#1}{#2}}}
\newcommand{\highlight}[2]{\textcolor{#1}{\textbf{#2}}}
\newcommand{\td}[1]{{\setmainfont{Monaco.ttf}{#1}}}  % 測資、程式碼字體
% \newcommand{\td}[1]{{\begin{spacing}{1.2}\setmainfont{Monaco.ttf}{#1}\end{spacing}}}  % 測資、程式碼字體
\newcommand{\bd}[1]{{\setmainfont{DIN-BoldAlternate Regular.otf}{\CJKfamily{\chfont}#1}}}  % 題目字體

\newenvironment{tests}
    {\VerbatimEnvironment\setlength\parindent{0pt} \setmainfont{Monaco.ttf} \fontsize{11}{11}\selectfont\begin{Verbatim}}% \begin{tests}
    {\end{Verbatim}\par\normalsize}% \end{tests}

\newenvironment{mycenter}[1][\topsep]
    {\setlength{\topsep}{#1}\par\kern\topsep\centering}% \begin{mycenter}[<len>]
    {\par\kern\topsep}% \end{mycenter}

\newcommand{\ContestName}{$111$ 學年度板橋高中資訊學科能力競賽 Round $1$}
\newcommand{\Logo}{\includegraphics[width=2.4cm]{logo.png}}

\newcommand\AtPageUpperRight[1]{\AtPageUpperLeft{\put(\LenToUnit{0.83\paperwidth},\LenToUnit{-2.7cm}){#1}}}

\newcommand{\newblank}{\newpage\begin{center}\hspace{0pt}\vfill{\fontfamily{cmr}\selectfont\textit{\Large{This page is intentionally left blank.}}}\vfill\hspace{0pt}\end{center}}

%%% 重定義一些 command %%%
\newcommand{\N}{\mathbb{N}}
\newcommand{\Z}{\mathbb{Z}}
\newcommand{\Q}{\mathbb{Q}}
\newcommand{\R}{\mathbb{R}}
\renewcommand{\C}{\mathbb{C}}
\newcommand{\F}{\mathbb{F}}
\renewcommand{\O}{\mathcal{O}}
\newcommand{\floor}[1]{\left\lfloor{#1}\right\rfloor}
\newcommand{\ceil}[1]{\left\lceil{#1}\right\rceil}
\newcommand{\link}[2]{\href{#1}{\textcolor{cyan}{\underline{#2}}}}

\pagestyle{fancy}
\lhead{\ContestName}
\renewcommand{\headrulewidth}{0pt}
\AddToShipoutPictureBG{\AtPageUpperRight{\put(20, -5)\Logo}}
\renewcommand{\arraystretch}{0.8}

\definecolor{codegreen}{rgb}{0,0.6,0}
\definecolor{codegray}{rgb}{0.5,0.5,0.5}
\definecolor{codebrown}{rgb}{0.56, 0.08, 0.0}
\definecolor{backcolour}{rgb}{0.95,0.95,0.92}

\lstdefinestyle{code}{
%    backgroundcolor = \color{backcolour},
    commentstyle = \color{codegreen},
    keywordstyle = \color{codebrown},
    numberstyle = \tiny\color{codegray},
    stringstyle = \color{codepurple},
    basicstyle = \ttfamily\footnotesize,
    breakatwhitespace = false,
    breaklines = true,
    captionpos = b,
%    frame = single,
    keepspaces = true,
    showspaces = false,
    showstringspaces = false,
    showtabs = false,
    tabsize = 4
}

\lstset{style = code}
