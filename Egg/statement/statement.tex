\documentclass[10pt, a4paper]{article}
\usepackage[top=3cm, bottom=3cm, left=2cm, right=2cm]{geometry} % 上下左右距離邊緣 2cm
\usepackage{amsmath, amsthm, amssymb}   % AMS 數學環境
\usepackage{enumitem}   % enumerate itemize
\usepackage{xeCJK}      % 中文
\usepackage{fontspec}   % 設定字體
\usepackage{anyfontsize}    % \fontsize{}{}
\usepackage{graphicx}   % 圖片
\usepackage{caption}    % for \caption*
\usepackage{fancyhdr}   % 頁首頁尾
\usepackage{titlesec}   % 設定 section 字體、\titlespacing
\usepackage{tabularx, makecell} % 加強版表格
\usepackage{hyperref}   % 超連結
\usepackage{xcolor}     % 文字顏色
\usepackage{ulem}       % 底線
\usepackage{eso-pic}    % 圖標
\usepackage{booktabs}   % 表格線條
\usepackage{listings}   % 程式碼上色
\usepackage{indentfirst}    % 縮排
\usepackage{setspace}   % 行間距
\usepackage{seqsplit}   % 換行
\usepackage{fancyvrb}
\usepackage[T1]{fontenc}

% 設定 section, subsection 字體
\titleformat*{\section}{\fontsize{15.5}{15.5}\selectfont\bfseries}
\titleformat*{\subsection}{\fontsize{13}{13}\selectfont\bfseries}

% 設定 math mode 字體
\DeclareMathSizes{10}{12}{7}{5}
\setlength{\mathsurround}{1.0pt}

\XeTeXlinebreaklocale "zh"  % 中文自動換行
\XeTeXlinebreakskip = 0pt plus 1pt  % 文字的彈性間距
\linespread{1.35}\selectfont  % 行距
\setlength{\parskip}{1.8ex} % 段落間距
\setitemize[1]{itemsep=-4pt,partopsep=0pt,parsep=1ex,topsep=5pt} % itemize 間距
\setenumerate[1]{itemsep=0pt,partopsep=0pt,parsep=\parskip,topsep=5pt} % enumerate 間距
\setlength{\parindent}{2em} % 縮排

% 設定 section 左邊的 hspace 以及上下的 vspace
\titlespacing\section{0pt}{12pt plus 4pt minus 2pt}{10pt plus 2pt minus 2pt}
\titlespacing\subsection{0pt}{16pt plus 4pt minus 2pt}{10pt plus 2pt minus 2pt}

% 設定中文字型
\setCJKmainfont[AutoFakeBold=4]{Noto Serif CJK TC} % 一般中文字體
\newcommand{\chfont}{SourceHanSansTW-Bold.ttf} % 標題字體
\setCJKfamilyfont{\chfont}{\chfont}[AutoFakeBold = 3]

% 設定英文字型
% \setmainfont{Times New Roman}
% \usepackage{newtxtext}
\setmonofont{Monaco.ttf}

\newcommand{\myhref}[2]{\textcolor{blue}{\href{#1}{#2}}}
\newcommand{\highlight}[2]{\textcolor{#1}{\textbf{#2}}}
\newcommand{\td}[1]{{\setmainfont{Monaco.ttf}{#1}}}  % 測資、程式碼字體
% \newcommand{\td}[1]{{\begin{spacing}{1.2}\setmainfont{Monaco.ttf}{#1}\end{spacing}}}  % 測資、程式碼字體
\newcommand{\bd}[1]{{\setmainfont{DIN-BoldAlternate Regular.otf}{\CJKfamily{\chfont}#1}}}  % 題目字體

\newenvironment{tests}
    {\VerbatimEnvironment\setlength\parindent{0pt} \setmainfont{Monaco.ttf} \fontsize{11}{11}\selectfont\begin{Verbatim}}% \begin{tests}
    {\end{Verbatim}\par\normalsize}% \end{tests}

\newenvironment{mycenter}[1][\topsep]
    {\setlength{\topsep}{#1}\par\kern\topsep\centering}% \begin{mycenter}[<len>]
    {\par\kern\topsep}% \end{mycenter}

\newcommand{\ContestName}{$111$ 學年度板橋高中資訊學科能力競賽 Round $1$}
\newcommand{\Logo}{\includegraphics[width=2.4cm]{logo.png}}

\newcommand\AtPageUpperRight[1]{\AtPageUpperLeft{\put(\LenToUnit{0.83\paperwidth},\LenToUnit{-2.7cm}){#1}}}

\newcommand{\newblank}{\newpage\begin{center}\hspace{0pt}\vfill{\fontfamily{cmr}\selectfont\textit{\Large{This page is intentionally left blank.}}}\vfill\hspace{0pt}\end{center}}

%%% 重定義一些 command %%%
\newcommand{\N}{\mathbb{N}}
\newcommand{\Z}{\mathbb{Z}}
\newcommand{\Q}{\mathbb{Q}}
\newcommand{\R}{\mathbb{R}}
\renewcommand{\C}{\mathbb{C}}
\newcommand{\F}{\mathbb{F}}
\renewcommand{\O}{\mathcal{O}}
\newcommand{\floor}[1]{\left\lfloor{#1}\right\rfloor}
\newcommand{\ceil}[1]{\left\lceil{#1}\right\rceil}
\newcommand{\link}[2]{\href{#1}{\textcolor{cyan}{\underline{#2}}}}

\pagestyle{fancy}
\lhead{\ContestName}
\renewcommand{\headrulewidth}{0pt}
\AddToShipoutPictureBG{\AtPageUpperRight{\put(20, -5)\Logo}}
\renewcommand{\arraystretch}{0.8}

\definecolor{codegreen}{rgb}{0,0.6,0}
\definecolor{codegray}{rgb}{0.5,0.5,0.5}
\definecolor{codebrown}{rgb}{0.56, 0.08, 0.0}
\definecolor{backcolour}{rgb}{0.95,0.95,0.92}

\lstdefinestyle{code}{
%    backgroundcolor = \color{backcolour},
    commentstyle = \color{codegreen},
    keywordstyle = \color{codebrown},
    numberstyle = \tiny\color{codegray},
    stringstyle = \color{codepurple},
    basicstyle = \ttfamily\footnotesize,
    breakatwhitespace = false,
    breaklines = true,
    captionpos = b,
%    frame = single,
    keepspaces = true,
    showspaces = false,
    showstringspaces = false,
    showtabs = false,
    tabsize = 4
}

\lstset{style = code}


\begin{document}
\section*{\bd{第三題:丟雞蛋問題(Egg) \color{blue}{[此題為互動題 Interactive]}}}
\subsection*{\bd{問題敘述}}
丟雞蛋問題是一個經典的題目,你現在正身處於一棟 $M$ 層樓高的大樓,手上有 $60$ 顆一模一樣的雞蛋,已知這些雞蛋只要在\bd{\color{red}{超過 $h$ 樓}}的地方將其摔出就會破裂;不超過則不會。但由於我們目前還不知道 $h$ 是多少,請你在摔完這些雞蛋之前,回答出 $h$ 的值。
\subsection*{\bd{實作細節}}
\subsubsection*{\bd{C++}}

你需要在首行加入 \texttt{\#include "Egg.h"},並完成以下函式:

\begin{lstlisting}[language = C++]
long long height_limit(long long M);
\end{lstlisting}

\begin{itemize}
    \item $M$ 代表 $h$ 的範圍介於 $[1,M]$,其中 $h$ 為整數, $1\le M\le 10^{18}$。
    \item 該函式需要在被呼叫後,回傳 $h$ 的正確數值。
\end{itemize}

你的程式可以呼叫以下函式:

\begin{lstlisting}[language = C++]
int is_broken(long long k);
\end{lstlisting}

\begin{itemize}
    \item 代表你要將一顆雞蛋從高度 $k$ 的樓層摔下去,\texttt{is\_broken($k$)} 會回傳 $1$ 或 $0$,如果回傳 $1$ 代表雞蛋破了;$0$ 則代表雞蛋沒破。$k$ 必須是一個介於 $[1,M]$ 的整數。
    \item 該函式只能被呼叫 $60$ 次,即使雞蛋沒破,你也無法將其回收。
\end{itemize}

如果不符合上述條件限制,你的程式會被判為 \bd{Wrong Answer};否則你的程式會被判斷為 \bd{Accepted}。

\subsubsection*{\bd{Python3}}

你需要完成以下函式:

\begin{lstlisting}[language = python]
def height_limit(M: int) -> int
\end{lstlisting}

\begin{itemize}
    \item $M$ 代表 $h$ 的範圍介於 $[1,M]$,其中 $h$ 為整數, $1\le M\le 10^{18}$。
    \item 該函式需要在被呼叫後,回傳 $h$ 的正確數值。
\end{itemize}

你的程式可以呼叫以下函式:

\begin{lstlisting}[language = python]
def is_broken(k: int) -> int
\end{lstlisting}

\begin{itemize}
    \item 代表你要將一顆雞蛋從高度 $k$ 的樓層摔下去,\texttt{is\_broken($k$)} 會回傳 $1$ 或 $0$,如果回傳 $1$ 代表雞蛋破了;$0$ 則代表雞蛋沒破。$k$ 必須是一個介於 $[1,M]$ 的整數。
    \item 該函式只能被呼叫 $60$ 次,即使雞蛋沒破,你也無法將其回收。
    \item 請在 \texttt{height\_limit} 函式內的第一行加入 \texttt{from \_\_main\_\_ import is\_broken} 來導入此函式,才能做使用。如:

\begin{lstlisting}[language = python]
def height_limit(M: int) -> int:
    from __main__ import is_broken
\end{lstlisting}
\end{itemize}

如果不符合上述條件限制,你的程式會被判為 \bd{Wrong Answer};否則你的程式會被判斷為 \bd{Accepted}。

\subsection*{\bd{互動範例}}
考慮以下的測試資料: $M=10, h=4$。

一個被評分程式判斷為 \bd{Accepted} 的互動例子顯示如下:
\begin{center}
    \begin{tabular}[t]{@{}ll@{}}
    \toprule
    評分程式端 & 參賽者端\\
    \midrule
    呼叫 \texttt{height\_limit($10$)} & \\
    回傳 $0$ & 呼叫 \texttt{is\_borken($1$)}\\
    回傳 $0$ & 呼叫 \texttt{is\_borken($2$)}\\
    回傳 $0$ & 呼叫 \texttt{is\_borken($3$)}\\
    回傳 $0$ & 呼叫 \texttt{is\_borken($4$)}\\
    回傳 $1$ & 呼叫 \texttt{is\_borken($5$)}\\
    & 回傳 $4$\\
    \bottomrule
    \end{tabular}
\end{center}
\subsection*{\bd{評分說明}}
本題共有 $4$ 組測試題組,條件限制如下所示。每一組可有一或多筆測試資料,該組所有測試資料皆需答對才會獲得該組分數。
\begin{center}
    \begin{tabular}[t]{@{}ccl@{}}
    \toprule
    子任務 & 分數 & 額外輸入限制\\
    \midrule
    0 & 0  & 互動範例($M=10, h=4$)。\\
    1 & 12 & $M=2$。\\
    2 & 20 & $M=60$。\\
    3 & 68 & 無額外限制。\\
    \bottomrule
    \end{tabular}
\end{center}

\newpage

\subsection*{\bd{範例評分程式}}
範例評分程式以下列格式讀取輸入:
\begin{itemize}
    \item 第一列: $M\;h$
\end{itemize}
其中 $M,h$ 如題目所述。\\

\bd{請注意:使用自己上傳的測試資料進行測試時,沒有下面 MSG 描述的情形時你會得到 Accepted。}如果你的程式被評為 \bd{Accepted},範例評分程式會輸出 Accepted: $q$,其中 $q$ 表示呼叫 \texttt{is\_broken} 函式的總次數。如果你的程式被評為 \bd{Wrong Answer},範例評分程式會輸出 Wrong Answer: MSG,其中 MSG 格式與意義如下:
\begin{itemize}
    \item invalid broken query: 不合法的 \texttt{is\_broken} 呼叫。
    \item too many queries: 呼叫上述函式的總次數超過 $60$ 次。
    \item incorrect height: \texttt{height\_limit} 回傳的值與題目的 $h$ 不相符。
\end{itemize}

在 CMS 內的附件檔案中,有一個名為「Egg.zip」的壓縮檔,下載後解壓縮可以找到三個資料夾「cpp」、「py」和「examples」,資料夾的意義分別為:
\begin{itemize}
    \item cpp: 內部包含一個檔案「Egg.cpp」,你可以直接上傳該檔案來獲得 12 分,並且往後你要寫的程式都可以參考這份檔案,並在這份檔案內進行修改、編譯及執行。

\bd{請注意,檔案內有兩行註解分別為「do not modify above」和「do not modify below」,這兩行意味著希望你盡量只更動被這兩行夾住的區域,若你對互動題並沒有很熟悉,請不要更動外面的區域,更動外面的區域不會也不可能讓你能直接拿到更高的分數。}

基本上你不需要更動除此之外的檔案,但若有需求可以在理解運作原理後進行更動。

    \item py: 內部包含一個檔案「Egg.py」,你可以直接上傳該檔案來獲得 12 分,並且往後你要寫的程式都可以參考這份檔案,並在這份檔案內進行修改。

若要執行程式,請切換到另一個檔案「grader.py」,直接執行後「grader.py」會嘗試與你修改過的「Egg.py」做連結後開始運行。\bd{請盡量只更動「Egg.py」,若你對互動題並沒有很熟悉,請不要更動「grader.py」,更動「grader.py」不會也不可能讓你能直接拿到更高的分數。}

基本上你不需要更動除此之外的檔案,但若有需求可以在理解運作原理後進行更動。

    \item examples: 內部包含互動範例的輸入和合法輸出。
\end{itemize}

\bd{請不要嘗試撰寫題目指定需要函式以外的任何東西,例如自行輸入、輸出等,無論是「grader.cpp」還是「grader.py」都僅供參考用,並與 Judge 上的有所落差。若有疑似偷取資料的行為,將視為作弊,嚴重者可能會取消資格並以校規處分。}

\end{document}
